\section*{Download Source}

Right now you must download the source by cloning the git repository from \href{http://code.google.com/p/keycpp/source/checkout}{\tt code.\-google.\-com/p/keycpp/}. In the future as the library matures, there will be a compressed file to download and possibly a package for Debian/\-Ubuntu.

\subsection*{Dependencies}

Currently this project makes use of the \href{http://www.sourceforge.net/projects/kissfft/}{\tt Kiss F\-F\-T} and \href{http://code.google.com/p/gnuplot-cpp/}{\tt gnuplot-\/cpp} open source projects as well as the \href{http://www.netlib.org/lapack/}{\tt L\-A\-P\-A\-C\-K} libraries, \href{http://www.gnuplot.info/}{\tt Gnuplot} plotting program, and the \href{http://www.boost.org/doc/libs/1_54_0/libs/numeric/odeint/doc/html/index.html}{\tt Boost odeint} library. The sources from \href{http://www.sourceforge.net/projects/kissfft/}{\tt Kiss F\-F\-T} and \href{http://code.google.com/p/gnuplot-cpp/}{\tt gnuplot-\/cpp} have been incorporated into this project.

The {\itshape only} extra dependencies that you need on your system are the \href{http://www.netlib.org/lapack/}{\tt L\-A\-P\-A\-C\-K} libraries, \href{http://www.gnuplot.info/}{\tt Gnuplot}, and the \href{http://www.boost.org/doc/libs/1_54_0/libs/numeric/odeint/doc/html/index.html}{\tt Boost odeint} library.

\subsubsection*{Ubuntu (and various \href{https://wiki.ubuntu.com/UbuntuFlavors}{\tt flavors})}

{\itshape N\-O\-T\-E\-: For permformance reasons, use of Open\-B\-L\-A\-S/\-L\-A\-P\-A\-C\-K is preferred over the default versions. Also, for Intel C\-P\-U's, Intel's M\-K\-L library provides the best performance.}

To acquire all the required dependencies you can execute the following commands\-:

{\ttfamily sudo apt-\/get install build-\/essential}

Until Ubuntu 13.\-10 comes out in October 2013, you need to install \href{http://headmyshoulder.github.io/odeint-v2/}{\tt odeint} from their website. The way I did it was to install Boost (see below) and then copy the odeint header and source files into the appropriate Boost folders. This is because odeint first appears in Boost 1.\-53 and the newest version in the repository for Ubuntu 13.\-04 is 1.\-49.

{\ttfamily sudo apt-\/get install libboost-\/all-\/dev}

{\ttfamily sudo apt-\/get install libopenblas-\/dev}

{\ttfamily sudo apt-\/get install liblapack-\/dev}

\subsubsection*{Other Operating Systems}

Your mileage may vary.

\subsection*{Installation \& Usage}

{\itshape This library uses features only available in the \href{https://en.wikipedia.org/wiki/C%2B%2B11}{\tt C++11 standard}. Y\-O\-U M\-U\-S\-T \href{http://gcc.gnu.org/projects/cxx0x.html}{\tt C\-O\-M\-P\-I\-L\-E} W\-I\-T\-H T\-H\-I\-S S\-T\-A\-N\-D\-A\-R\-D.}

To install Key\-Cpp onto your system first clone the git repository to a directory of your choice\-: {\ttfamily git clone \href{https://code.google.com/p/keycpp/}{\tt https\-://code.\-google.\-com/p/keycpp/}}

The following command will compile the Key\-Cpp library and provide links to the library and header files in Ubuntu's default location\-:

{\ttfamily sudo ./\-I\-N\-S\-T\-A\-L\-L}

To uninstall the Key\-Cpp library, use the following command\-:

{\ttfamily sudo ./\-U\-N\-I\-N\-S\-T\-A\-L\-L}

If everything was successful, you should be able to compile and run the example program\-:

{\ttfamily cd examples}

{\ttfamily g++ -\/c -\/o obj/example.\-o example.\-cpp -\/std=c++11}

{\ttfamily g++ -\/o example obj/example.\-o -\/std=c++11 -\/lkeycpp -\/lblas -\/llapack}

{\ttfamily ./example}

If you are using Intel's M\-K\-L libraries, use this command instead of the one above\-: (where {\ttfamily //opt/intel/composer\-\_\-xe\-\_\-2013.4.\-183/mkl/lib/intel64} is the path of your M\-K\-L library {\ttfamily libmkl\-\_\-rt.\-so})

{\ttfamily g++ -\/o example obj/example.\-o -\/std=c++11 -\/lkeycpp -\/\-L/\textbackslash{}/opt/intel/composer\-\_\-xe\-\_\-2013.4.\-183/mkl/lib/intel64 -\/lmkl\-\_\-rt}

Alternatively, you could just type {\ttfamily make lib=lapack} or {\ttfamily make lib=mkl} inside the {\ttfamily examples} directory. Then type {\ttfamily ./example} to run the example code.

When you are writing your own programs be sure to link with the {\ttfamily libkeycpp.\-a}, {\ttfamily libblas.\-a}, and {\ttfamily liblapack.\-a} libraries (or {\ttfamily libmkl\-\_\-rt.\-so} if you are using Intel M\-K\-L). With {\ttfamily g++} the form is the same as used above. {\itshape D\-O N\-O\-T F\-O\-R\-G\-E\-T T\-O C\-O\-M\-P\-I\-L\-E W\-I\-T\-H T\-H\-E C++11 S\-T\-A\-N\-D\-A\-R\-D!}

\section*{Example Code}

{\bfseries {\ttfamily example.\-cpp}} 
\begin{DoxyCodeInclude}
1 \textcolor{preprocessor}{#include <iostream>}
2 \textcolor{preprocessor}{#include <keycpp/keycpp.h>}
3 \textcolor{keyword}{using namespace }std;
4 \textcolor{keyword}{using namespace }keycpp;
5 
6 \textcolor{comment}{// Define a class for our ordinary differential equation that we will use later:}
7 \textcolor{keyword}{class }\hyperlink{class_ode_class}{OdeClass}
8 \{
9     \textcolor{keyword}{public}:
10         \textcolor{keywordtype}{void} operator()(\textcolor{keyword}{const} vector<double> &y,
11                         vector<double> &dy,
12                         \textcolor{keyword}{const} \textcolor{keywordtype}{double} t)
13         \{
14             dy[0] = y[1]*y[2];
15             dy[1] = -y[0]*y[2];
16             dy[2] = -0.51*y[0]*y[1];
17         \}
18 \};
19 
20 \textcolor{keywordtype}{int} main(\textcolor{keywordtype}{int} argc, \textcolor{keywordtype}{char}** argv)
21 \{
22     \textcolor{comment}{// First, lets create some data: y1 = t^2 and y2 = t^3}
23     vector<double> t = \hyperlink{namespacekeycpp_a4e8769de1f22713d3564350d53125b26}{linspace}(-2.0,2.0,100);
24     vector<double> y1 = \hyperlink{namespacekeycpp_ac1ff99e34619478096c271b38df1f3d7}{times}(t,t);
25     vector<double> y2 = \hyperlink{namespacekeycpp_ac1ff99e34619478096c271b38df1f3d7}{times}(t,\hyperlink{namespacekeycpp_ac1ff99e34619478096c271b38df1f3d7}{times}(t,t));
26 
27     \textcolor{comment}{// Now, lets plot the data we just created:}
28     \hyperlink{classkeycpp_1_1_figure}{Figure} h1;
29     h1.plot(t,y1,\textcolor{stringliteral}{"b-"},\textcolor{stringliteral}{"linewidth"},2);
30     h1.hold\_on();
31     h1.plot(t,y2,\textcolor{stringliteral}{"r--"},\textcolor{stringliteral}{"linewidth"},2);
32     h1.grid\_on();
33     h1.xlabel(\textcolor{stringliteral}{"t"});
34     h1.ylabel(\textcolor{stringliteral}{"y"});
35     h1.legend(\{\textcolor{stringliteral}{"y1 = t^2"},\textcolor{stringliteral}{"y2 = t^3"}\});
36     \textcolor{keyword}{set}(h1,\textcolor{stringliteral}{"fontsize"},14);
37 
38     \textcolor{comment}{// This is how to solve linear equations of the form Ax = b:}
39     \hyperlink{classkeycpp_1_1matrix}{matrix<double>} A = \{\{1.0, 2.0\},
40                         \{1.0,-1.0\}\};
41     vector<double> b = \{1.1,
42                         2.1\};
43     vector<double> x = linsolve(A,b);
44     \textcolor{comment}{// Print the result to the screen:}
45     \hyperlink{namespacekeycpp_a6a8a286886d48471685b18b7782f1e4a}{disp}(x);
46 
47     \textcolor{comment}{// Now lets do something a little more complicated, solve an}
48     \textcolor{comment}{// ordinary differential equation (ODE):}
49     \textcolor{comment}{// y(1)' = y(2)*y(3);}
50     \textcolor{comment}{// y(2)' = -y(1)*y(3);}
51     \textcolor{comment}{// y(3)' = 0.51*y(1)*y(2);}
52     \textcolor{comment}{// With initial conditions at t = 0: y(1) = 0; y(2) = 1; y(3) = 1;}
53     \hyperlink{class_ode_class}{OdeClass} myOde;
54     vector<double> t2 = \hyperlink{namespacekeycpp_a4e8769de1f22713d3564350d53125b26}{linspace}(0.0,12.0,100);
55     vector<double> ICs = \{0.0, 1.0, 1.0\};
56     \hyperlink{classkeycpp_1_1matrix}{matrix<double>} y = ode45(myOde, t2, ICs);
57     
58     \textcolor{comment}{// Now that we have solved the ODE, lets plot the results:}
59     \hyperlink{classkeycpp_1_1_figure}{Figure} h2;
60     h2.plot(t2,y.getCol(0),\textcolor{stringliteral}{"-"});
61     h2.hold\_on();
62     h2.plot(t2,y.getCol(1),\textcolor{stringliteral}{"-."});
63     h2.plot(t2,y.getCol(2),\textcolor{stringliteral}{"x"});
64     h2.xlabel(\textcolor{stringliteral}{"t"});
65     h2.ylabel(\textcolor{stringliteral}{"y"});
66     \textcolor{keyword}{set}(h2,\textcolor{stringliteral}{"fontsize"},14);
67     h2.title(\textcolor{stringliteral}{"ODE Solution"});
68     
69     \textcolor{keywordflow}{return} 0;
70 \}
\end{DoxyCodeInclude}
 \par
 {\bfseries {\ttfamily Text} Output} 
\begin{DoxyCodeInclude}
1 1.76667
2 -0.333333
\end{DoxyCodeInclude}
 \par
 {\bfseries {\ttfamily Plot} Output}  \par
  \par


\section*{M\-A\-T\-L\-A\-B/\-Octave to Key\-Cpp Conversion Chart}

Although the goal of this library is to offer a C++ interface similar in syntax to M\-A\-T\-L\-A\-B/\-Octave, there are some minor differences. The goal of this document is to provide a conversion chart for some of the most commonly used features.

Note\-: You can omit the {\ttfamily keycpp\-:\-:} prefix from the following commands by placing {\ttfamily using namespace keycpp;} in the same scope. This shortcut should be used with care as collisions with other libraries are possible.

\begin{TabularC}{3}
\hline
\rowcolor{lightgray}{\bf {\itshape M\-A\-T\-L\-A\-B/\-Octave} }&{\bf {\itshape Key\-Cpp} }&{\bf Notes }\\\cline{1-3}
{\ttfamily A(1,1)} &{\ttfamily A(0,0);} &Indexing starts at 0 in Key\-Cpp \\\cline{1-3}
{\ttfamily A(\-N,\-N)} &{\ttfamily A(N-\/1,N-\/1);} &\\\cline{1-3}
{\ttfamily size(\-A,1)} &{\ttfamily keycpp\-::size(\-A,1);} &\\\cline{1-3}
{\ttfamily size(\-A,2)} &{\ttfamily keycpp\-::size(\-A,2);} &\\\cline{1-3}
{\ttfamily A(\-:,k)} &{\ttfamily A.\-get\-Col(k-\/1);} &C++ restricts the use of {\ttfamily \-:} \\\cline{1-3}
{\ttfamily A(k,\-:)} &{\ttfamily A.\-get\-Row(k-\/1);} &\\\cline{1-3}
{\ttfamily A.'} &{\ttfamily keycpp\-::transpose(\-A);} &C++ does not allow overloading {\ttfamily .'} \\\cline{1-3}
{\ttfamily A'} &{\ttfamily keycpp\-::ctranspose(\-A);} &C++ does not allow overloading {\ttfamily '} \\\cline{1-3}
{\ttfamily A = zeros(m,n)} &{\ttfamily \hyperlink{classkeycpp_1_1matrix}{keycpp\-::matrix}$<$double$>$ A = \hyperlink{namespacekeycpp_a5699c522088657287bf0ac01173b716c}{keycpp\-::zeros}$<$double$>$(m,n);} &or more simply\-: {\ttfamily \hyperlink{classkeycpp_1_1matrix}{keycpp\-::matrix}$<$double$>$ A(m,n);} \\\cline{1-3}
{\ttfamily A = ones(m,n)} &{\ttfamily \hyperlink{classkeycpp_1_1matrix}{keycpp\-::matrix}$<$double$>$ A = \hyperlink{namespacekeycpp_a388f91a0ccf34978ef9403ccd0c680bf}{keycpp\-::ones}$<$double$>$(m,n);} &\\\cline{1-3}
{\ttfamily A.$\ast$\-B} &{\ttfamily keycpp\-::times(\-A,\-B);} &C++ does not allow overloading {\ttfamily .$\ast$} or {\ttfamily ./} \\\cline{1-3}
{\ttfamily A./\-B} &{\ttfamily keycpp\-::rdivide(\-A,\-B);} &\\\cline{1-3}
{\ttfamily A\textbackslash{}b} &{\ttfamily keycpp\-::linsolve(\-A,b);} &{\ttfamily b} is a vector \\\cline{1-3}
{\ttfamily \mbox{[}V, D\mbox{]} = eig(\-A,\-B)} &{\ttfamily std\-::vector$<$std\-::complex$<$double$>$$>$ d = keycpp\-::eig(\-A,\-B,\&\-V);} &Non-\/\-Hermitian generalized eigenvalue/eigenvector solver uses L\-A\-P\-A\-C\-K. \\\cline{1-3}
{\ttfamily x = linspace(0,10,\-N\-\_\-x)} &{\ttfamily std\-::vector$<$double$>$ x = \hyperlink{namespacekeycpp_a4e8769de1f22713d3564350d53125b26}{keycpp\-::linspace}(0.\-0,10.\-0,N\-\_\-x);} &\\\cline{1-3}
{\ttfamily x = logspace(1,3,\-N\-\_\-x)} &{\ttfamily std\-::vector$<$double$>$ x = \hyperlink{namespacekeycpp_a1704b4adc18c9353ee63fa63539df54d}{keycpp\-::logspace}(1.\-0,3.\-0,N\-\_\-x);} &{\ttfamily 10 $<$= x $<$= 1000} \\\cline{1-3}
{\ttfamily A = diag(\mbox{[}a1, a2, a3\mbox{]})} &{\ttfamily \hyperlink{classkeycpp_1_1matrix}{keycpp\-::matrix}$<$double$>$ A = keycpp\-::diag(\{a1, a2, a3\});} &{\ttfamily a1}, {\ttfamily a2}, and {\ttfamily a3} are scalar elements of {\ttfamily A} \\\cline{1-3}
{\ttfamily A = \mbox{[}\mbox{[}a1, a2\mbox{]}; \mbox{[}a3, a4\mbox{]}\mbox{]}} &{\ttfamily \hyperlink{classkeycpp_1_1matrix}{keycpp\-::matrix}$<$double$>$ A = \{\{a1, a2\}, \{a3, a4\}\};} &\\\cline{1-3}
\end{TabularC}
